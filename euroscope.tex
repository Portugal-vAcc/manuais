\documentclass[10pt]{report}
\usepackage[margin=0.9in]{geometry}
\usepackage{fontspec}
\usepackage[explicit]{titlesec}
\usepackage{setspace}
\usepackage{enumitem}
\usepackage{changepage}
\usepackage{paracol}
\usepackage{url}
\usepackage{polyglossia}
\usepackage{hyperref}

% polyglossia
\setmainlanguage{portuges}

% fontspec
\setmainfont{FreeSerif}
\setsansfont{FreeSans}
\setmonofont{FreeMono}

% titlesec
\titleformat{\chapter}[display]{\onehalfspacing\bfseries\centering}{\Large \chaptertitlename~\thechapter}{1.5em}{\Large\MakeUppercase{#1}}
\titleformat{\section}[display]{\bfseries\normalsize\centering}{\thesection\ \ \ \ \MakeUppercase{#1}}{-1em}{}
\titleformat{\subsection}[display]{\bfseries\normalsize\centering}{\thesubsection\ \ \ \ #1}{-1em}{}
\titleformat{\paragraph}[runin]{\scshape\bfseries}{}{1em}{#1}

\newenvironment{circumstances}{
    \titleformat{\subsection}[display]{\normalsize}{\thesubsection\ \ \ \ \MakeUppercase{##1}}{-1em}{}
    \titleformat{\paragraph}[runin]{\scshape}{}{1em}{}

    \begin{paracol}{2}
        \setlength{\columnseprule}{0.4pt}
        \setlength{\columnsep}{2em}
} {
    \end{paracol}
}

% table of contents
\setcounter{tocdepth}{3}
\setcounter{secnumdepth}{6}
\setcounter{chapter}{0}

\begin{document}

\title{Euroscope Manual}
\author{Portugal vACC}

\maketitle

\tableofcontents

\chapter*{Prefácio}

\paragraph{} Este documento pretende explicar alguns detalhes em relação ao pacote de LPPC,
nomeadamente, os ficheiros de vistas radar, as TAGs, e as novidades na área \textit{Display
Settings}.

\paragraph{} Antes de mais, o pacote está criado de forma a ser bastante simples abri-lo. Bastará
arrastar, toda, a pasta para a pasta \path{%userprofile%\Documents\Euroscope}. Depois será
unicamente necessário abrir o Euroscope como sempre, tendo em atenção que se utiliza o novo
\path{LPPC_CTR.prf}.

\paragraph{} De notar que na primeira vez será necessário configurar o seguinte:
\begin{itemize}
    \item Posição de controlo (escolher no menu drop-down) e servidor;
    \item Nome, senha e rating do CTA;
    \item Microfone, auscultador e botão PTT (Push-To-Talk).
\end{itemize}

\chapter{Vistas radar}

\paragraph{} Estes encontram-se na pasta \path{%userprofile%\Documents\Euroscope\LPPC\ASR}. Cada
\path{.asr} é uma vista de radar, com definições próprias (TAG activa, zoom e centro de
visualização do radar, definições de \textit{Display Settings}).

\paragraph{} Cada ficheiro tem o seu propósito, e dá para o descortinar no nome do mesmo. Por
exemplo, \path{LPPC_CTR} será para controlar a posição de Lisboa Control, e começará com uma vista
a abranger todo o espaço aéreo da RIV de Lisboa. Por outro lado, \path{LPPT_APP} focará unicamente
a zona de aproximação da TMA de Lisboa, enquanto que, por exemplo, \path{LPPT_GND} mostrará, por
default, o aeroporto de Lisboa.

\paragraph{} Pode-se abrir um novo ficheiro \path{.asr} indo a \textit{OPEN SCT} – \textit{Open},
escolhendo de seguida o ficheiro correspondente à vista que se pretende abrir. É possível ter-se
várias vistas abertas, e trocar de uma para a outra utilizando a tecla \textbf{F7}. É também
possível criar em \textit{OTHER SET} – \textit{General Settings} – \textit{Page 2} atalhos para
abrir vistas específicas, evitando assim ter de ir a \textit{OPEN SCT} – \textit{Open}.

\paragraph{} Para, rapidamente, se voltar à vista inicial basta carregar simultaneamente nas teclas
\textbf{CTRL} e \textbf{HOME}.

\chapter{TAGs}

\paragraph{} Para além da TAG default (\textit{Matias (built in)}) foram adicionadas mais quatro
TAGs, cada uma com funções e características distintas. Esta quatro TAGs são réplicas das TAGs
utilizadas no controlo real, com algumas adições para facilitar o controlo no Euroscope.

\paragraph{} Nos sub-capítulos abaixo vai-se abordar estas quatro TAGs, e dizer em qual posição é
recomendada a sua utilização – obviamente nada proíbe o CTA de utilizar outra TAG, ou de usar a TAG
default, ou uma por ele próprio criada.

\paragraph{} As informações presentes na TAG vão ser enumeradas da esquerda para a direita e de
cima para baixo.

\section{PT vACC – NORM}
\label{sec:tag-norm}
\paragraph{} Nesta TAG conseguimos, sem ser necessário colocar-lhe o rato em cima, o indicativo de
chamada da aeronave, o controlador seguinte, a altitude actual, a altitude temporário (atribuída
pelo CTA) e a velocidade (ground speed).

\paragraph{} Colocando-se o rato por cima, abrindo portanto a versão detalhada da TAG é ainda
adicionado o COPX para coordenar directos, a informação de rumo e velocidade instruída pelo CTA,
caixa de texto livre, e o código transponder.

\paragraph{}
Carregando com o botão direito do rato na velocidade é aberto o Menu para atribuir velocidade em
Mach.

\paragraph{} Esta TAG é recomendada para ser utilizada pelas posições de Aproximação e Torre.

\section{PT vACC – RDUC}
\paragraph{} Esta TAG é em tudo igual à TAG \hyperref[sec:tag-norm]{PT vACC – NORM}, com a único
excepção de a TAG quando não está detalhada (i.e., quando não se está a passar com o rato em cima
dela) não mostrar a velocidade.

\paragraph{} Esta TAG é recomendada para ser utilizado pelas posições de Centro.


\section{PT vACC – GND RDR}
\paragraph{} Esta TAG mostra na versão não detalhada unicamente o indicativo de chamada da
aeronave.

\paragraph{} Após duplo-clique na TAG (“abrindo” a TAG) consegue-se ver o tipo de aeronave passando
com o rato por cima da mesma.

\paragraph{} Esta TAG é recomendada para ser utilizado pelas posições de Ground.


\section{PT vACC – XPDR}
\paragraph{} Esta TAG na versão não detalhada é igual à TAG “PT vACC – NORM”, com a única diferença
ser mostrar ainda o código transponder.

\paragraph{} A versão detalhada é igual às TAGs “PT vACC – NORM” e “PT vACC – RDUC”.

\paragraph{} Esta TAG geralmente não é utilizada, mas ficará ao critério do CTA dar-lhe uso se e
quando o achar conveniente.


\chapter{Display Settings}
\paragraph{} Como explicado no capítulo das vistas radar cada uma destas vem com pré-definida com o
que está visível. Porém, há bastantes mais coisas que poderão ser activadas caso o CTA assim o
queira. Antes de mais, todas as secções a que este documento a partir de agora se vai referir
encontram-se em “Display Settings” (para abrir: OTHER SET – Display Settings).

\paragraph{} Nas primeiras 5 secções (VOR a Fixes) não há nenhum detalhe diferente do normal,
portanto serão saltadas.

\paragraph{} Porém, a secção “Stars” já contém algumas particularidades. Para além dos desenhos das
aproximações que já conhecemos, pode-se aqui (des)activar o desenho de padrões de espera na vista
radar. Por pré-definição estão visíveis 9 padrões de espera (ADSAD, EKMAR, RINOR, UMUPI em LPPT;
GIMAL em LPFR; DIVUT, RETMO em LPPR; ABUSU, FUSUL em LPMA).

\paragraph{} Nas secções seguintes volta a não haver nada de novo que mereça referência, até
chegarmos à secção “ARTCC boundary”. Aqui é possível activar o desenho das áreas restritas.. Por
pré-definição estão todas desactivadas. Para além das áreas restritas é possível (des)activar o
desenho dos diferentes sectores da RIV de Lisboa. No total há 6 (NORTH, CENTRE, SOUTH, DEMOS,
VERAM, MADEIRA), sendo que cada um destes se divide em “Upper sector” e “Lower sector”. Para
reduzir a quantidade de opções, os sectores DEMOS, VERAM e MADEIRA são só os sectores superiores,
enquanto que os outros três sectores têm as duas opções. O sector inferior vai até altura do solo
até nível de voo 245, enquanto que o superior começa em 245 e não tem limite superior. A única
diferença entre os sectores inferiores e superiores existe na fronteira com Espanha. Abaixo de
nível de voo 245 o tráfego sai de controlo Português assim que atravessar a fronteira entre os dois
países, acima de 245 os limites entre as duas regiões são uma linha recta a Norte e outra a Oeste,
que a Norte passa por, por exemplo, PINEK e ABUPI, e a Oeste por TOSDI e OGERO.

\paragraph{} Para além destes sectores foram ainda criados os sectores LPPC East e LPPC West. Cada
um deles junta três sectores (LPPC East aglomera os sectores DEMOS, VERAM e MADEIRA, enquanto que
LPPC West contém os sectores NORTH, CENTRE e SOUTH [todos eles os sectores superiores]).

\paragraph{} A diferença entre activar os sectores East/West ao invés de cada sector individual é o
primeiro ter unicamente as linhas de contorno da RIV, e não as linhas a dividirem cada sector na
horizontal.

\paragraph{} Por pré-definição estão ligados os sectores LPPC East e LPPC West.

\paragraph{} Na secção seguinte, ARTCC low boundary, pode ser (des)activado o desenho da
TMA/MRVA/etc. de cada aeroporto. Por pré-definição estão todos ligados. Nesta secção encontram-se
também os túneis VFR na TMA de Lisboa, que incialmente estão todos desactivados.

\paragraph{} A secção GEO não contém nenhuma particularidade, portanto passemos à secção Free Text.
Aqui há várias detalhes novos: “LPMA MRVA”, “LPPR MRVA”, LPPT MRVA” inserem na vista radar as
altitudes de cada zona da MRVA. Em SCT2 é possível activar a identificação das áreas restritas. Por
limitações do Euroscope foi necessário ficar numa sub-secção com um nome genérico. A parte
“DynPoints” deverá ser ignorada e mantida desactivada. Estes pontos são utilizados para alguns
desenhos, mas os pontos em si não nos serão úteis para controlar.

\paragraph{} As altitudes das áreas restritas podem ser consultadas aqui:
\url{https://www.nav.pt/CD/NAVProtectedData/eAIP/html/eAIP/LP-ENR-5.1-en-PT.html#ENR-5.1} (requer login
no AIP da NAV).


\end{document}
